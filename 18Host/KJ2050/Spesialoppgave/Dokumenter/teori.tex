\section{Teori}

\subsection{Redokstitrering og Jodometri}

I en redokstitrering reagerer stoffene som titreres mot hverandre i en redoksreaksjon\cite{analyticalchem2014}. Ekvivalenspunkt kan bestemmes ved bruk av farget oksidasjonsmiddel som kaliumpermanganat\cite{snlredoks}, eller ved bruk av en passende indikator for stoffene tilstede.

Jodometri er en form for redokstitrering hvor jod brukes som reduksjonsmiddel. Analytten reduseres i et overskudd av jodid. En mye brukt indikator i jodometri er stivelse.\cite{snljodometri} Selv ved meget små mengder jod i løsningen vil tilsetting av stivelse gi en sterk blåfarge\cite{snljod}. Reaksjonsligning for jodometrisk titrering av kobber er:

\begin{subequations}
	\begin{equation}
		\chemfig{2Cu^{2+}} + \chemfig{3I^{-}} \rightleftharpoons \chemfig{2Cu^{+}} + \chemfig{I_3^{-}}
	\end{equation}
	\begin{equation}
		\chemfig{2Cu^{2+}} + \chemfig{2I^{-}} \rightleftharpoons \chemfig{2CuI}
	\end{equation}
\end{subequations}

Den resulterende konsentrasjonen av jod (opptrer som \chemfig{I_3^{-}} i overskudd av jodid) kan bestemmes ved å titrere mot en standard av natriumtiosulfat (\chemfig{Na_2S_2O_3}). Kobberjodid felles ut av løsningen og oksideres ikke tilbake av tiosulfaten. Reaksjonligning for dette samt totalreaksjon for titreringen blir:

\begin{subequations}
	\begin{equation}
		\chemfig{I_3^{-}} + 2 \chemfig{S_2 O_3^{2-}} + 2 \chemfig{I^{-}} + \chemfig{S_4 O_6^{2-}}
	\end{equation}
	\begin{equation}
		2 \chemfig{Cu^{2+}} + 2 \chemfig{S_2 O_3^{2-}} + 2 \chemfig{I^{-}} \rightarrow
		2 \chemfig{CuI} + \chemfig{S_4 O_6^{2-}}
	\end{equation}
\end{subequations}

\subsubsection{Instilling av standard}

Natriumtiosulfat er ikke en god standardløsning og må instilles mot en primærstandard av kaliumjodat (\chemfig{K IO_3}). I surt miljø vil kaliumjodat reagere i overskudd av jodid til jod (trijodid) som tiosulfatstandarden kan titreres mot\cite{analyticalchem2014}. Reaksjonene for denne titreringen blir:

\begin{subequations}
	\begin{equation}
		\chemfig{IO_3^{-}} + 8\chemfig{I^{-}} + 6\chemfig{H^{+}} \rightleftharpoons 3\chemfig{I_3^{-}} + 3\chemfig{H_2O}
	\end{equation}
	\begin{equation}
		3\chemfig{I_3^{-}} + 6\chemfig{S_2O_3^{2-}} \rightleftharpoons 9\chemfig{I^{-}} + 3\chemfig{S_4O_6^{2-}}
	\end{equation}
	\label{eq:instill}
\end{subequations}

\subsubsection{Potensielle bireaksjoner ved jodometrisk titrering}

\chemfig{CO_2} er lettløselig i vann og danner karbonsyre \cite{snlCO2}, som vist i likning \ref{eq:CO2_i_vann}. Karbonsyra gir en svakt sur løsning ved at den spaltes og danner \chemfig{H^+}-ioner (likning \ref{syrelikevekt}). I surt miljø vil oksygen fra luft kunne redusere jodioner til jod, vist i likning \ref{O2_reduserer_jod}, derfor brukes \chemfig{CO_2}-fritt vann i jodometriske titreringer.

\begin{subequations}
	\begin{equation}
		\chemfig{CO_2} + \chemfig{H_2 O} \rightleftharpoons \chemfig{H_2 CO_3}
		\label{eq:CO2_i_vann}
	\end{equation}
	\begin{equation}
		\chemfig{H_2 CO_3} \rightleftharpoons \chemfig{H^{+}} + \chemfig{HCO_3^{-}}
		\label{syrelikevekt}
	\end{equation}
	\begin{equation}
		4 \chemfig{I^{-}} + \chemfig{O_2} + 4 \chemfig{H^{+}} \rightarrow 2 \chemfig{I_2} + 2 \chemfig{H_2 O}
		\label{O2_reduserer_jod}
	\end{equation}
\end{subequations}

Reaksjon \ref{hypojoditt} viser at jod i basisk løsning kan danne hypojoditt, \chemfig{HIO}. Dette stoffet er lite stabilt og spaltes raskt videre til jodid og jodat, som vist i likning \ref{eq:hypojodit}.\cite{snljodoksosyrer} For å unngå dannelse av hypojoditt må jodtitrering utføres med nøytral eller svakt sur løsning\cite{snljodimetri}.

\begin{subequations}
	\begin{equation}
		\chemfig{I_2} + \chemfig{OH^{-}} \rightleftharpoons \chemfig{IO^{-}} + \chemfig{I^{-}} + \chemfig{H^{+}}
		\label{hypojoditt}
	\end{equation}
	\begin{equation}
		3\chemfig{IO^{-}} \rightleftharpoons \chemfig{IO_3^{-}} + 2\chemfig{I^{-}}
	\end{equation}
	\label{eq:hypojodit}
\end{subequations}




\subsection{Utvidet spesialoppgave}

\subsubsection{Alternativ klassisk metode}


\subsubsection{Alternativ instrumentell metode}


\subsubsection{Alternative bruksområder for iodometrisk titrering}

