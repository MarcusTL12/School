

\subsection{Utvidet spesialoppgave}

\subsubsection{Alternativ klassisk metode: EDTA komplekstitrering}
\textit{En alternativ klassisk metode for bestemmelse av kobberioner er komplekstitrering med EDTA\cite{EDTACopper2}. Denne metoden bygger på det at EDTA danner chelatkomplekser med toverdige metallion som er veldig sterke komplekser. Denne kompleksdannelsen kan utnyttes til å finne konsentrasjonen av f.eks. kobber i en løsning da det er kjent at det dannes komplekser med støkiometrisk forhold 1:1. Dersom en har en indikator for enten kobber eller EDTA kan titreringen gjøres direkte\cite{EDTACopper}.}

\textit{Prosedyren for direkte EDTA titrering blir da: Vi har et kjent volum med analyttløsningen i en beholder med en indikator. Denne indikatoren må enten gi farge ved tilstedeværelse av kobber (kobberkomplekser er typisk blå) eller gi farge ved tilstedeværelse av EDTA. pH må holdes rundt 10 i løsningen da EDTA er en fireprotisk syre (etylendiamintetraeddiksyre) og vi trenger at det er i protolysesteg 4. Det er også viktig at pH ikke går for langt over 10 da kobber hydroksid er tungtløslig og ville felt ut ved høy pH. Egnet pH stabilisator vil da være en ammoniumbuffer da den har pKa i området rundt 10. EDTA løsning av kjent konsentrasjon tilsettes så til titrerendepunkt. Ved en indikator for kobber vil endepunktet sees ved at fargen fra indikatoren forsvinner da komplekseringen har brukt opp kobberet. Ved en indikator for EDTA vil en se en farge oppstå da det vil komme fritt EDTA i løsning når det ikke er mer kobber.}

\subsubsection{Alternativ instrumentell metode: ICP-MS}
\textit{En alternativ instrumentell metode for bestemmelse av kobberioner er ICP-MS\cite{ICPMSCopper}\cite{ICPMSCopper2}. ICP-MS (Inductively Coupled Plasma-Mass Spektroscopy) er en instrumentell metode for å finne den atomiske sammensetningen av en forbindelse/løsning/blanding. Den baserer seg på å komplet ionisere analytten til enkeltioner (derav plasma) som den akselerer ved elektromagnetiske felt og filtrerer med elektriske linser og kvadropoler som sentraliserer ionestrålen og detekterer masse/ladningsforholdet. Dette gir en ekstremt nøyaktig sammensetning av blandingen da vi nærmest teller enkeltatomer. Den er sensitiv helt ned i pico-femto miljø og er rimelig rask å utføre (rundt 15 minutter). Den kan imidlertid ikke gjøre alt og en av de viktigste begrensningene er at den kun gir statistikk på hvilke atomsammensetning du har (Empirisk formel) og ikke om hvilke kjemiske strukturer og sammensetninger en har. Dette gjør blant annet at organiske stoffer er ganske upraktisk å måle, da en bare får mye karbon, hydrogen, oksygen og nitrogen. Instrumentene er også forferdelig dyre, noe som gjør det litt mindre tilgjengelig for enhver analyse. Instrumentene får også problemer dersom det er for mye stoff i prøven. Om konsentrasjonene begynner å gå særlig mye over mikronivå kan en få problemer med stoffavsetninger på kjeglene som holder trykkdifferensialen i instrumentet. Disse har åpninger som er meget små for å slippe gjennom en tynn stråle med ioner og for å ha vakuum i detektordelen av instrumentet, som kan tettes til.}

\subsubsection{Alternative bruksområder for jodometrisk titrering}

\textit{Jodometri og kan brukes til å bestemme en rekke andre stoffer som kan reagere i redoks reaksjoner. Dette kan være andre monoatomiske metallioner som jern\cite{IodimetryIron} og tin\cite{IodometryTin} eller polyatomiske stoffer som kromater\cite{IodimetryChromate} og kalsiumoksider\cite{IodometryCalciumOx}.}
