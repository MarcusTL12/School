

\subsection{Utvidet spesialoppgave}

\subsubsection{Feilkilder for avvik fra fasit}
\textit{I forsøket er kaliumjodat brukt som primærstandard, som vil si at nesten alle målte verdier avhenger at nøyaktigheten til den standarden. Dette betyr at dersom standarden har tatt opp urenheter eller brutt ned over tid vil mengden frigjort jod til innstillingen av tiosulfat avvike noe fra den beregnede verdien fra utveid kaliumjodat\cite{IodStandardFeil}. Frigjøringen av jod fra kaliumjodat er også pH sensitiv\cite{IodometriLiberation}, noe som gjør feil pH til en feilkilde dersom denne ikke holdes stabil.}

\textit{I titreringen ble stivelse brukt som indikator. Denne brytes ned av jod over tid\cite{IodometryStarch} som reduserer mengden jod i løsningen som fører til undertitrering. Dette gjelder også mange andre organiske urenheter så derfor er urenheter og tidsbruk i titreringen feilkilder som fører til undertitrering.}


\subsubsection{Feilkilder for alternativ klassisk metode}
\textit{EDTA er ikke like utsatt for nedbryting som jod er, men komplekstitrering er fortsatt veldig utsatt for urenheter, spesielt for andre metallioner som for eksempel kalsium og magnesium\cite{EDTACalcium}\cite{EDTACalcium2} som begge er betydelige konsentrasjoner av i vann funnet i naturen. Dette er et problem da EDTA vil kompleksere med alt uselektivt og vil derfor bli tatt med i titreringen og gi et positivt avvik. Dette kan jobbes rundt for eksempel ved selektiv utfelling av urenhetene dersom det lar seg gjøre med tilgjengelige kjemikalier, men det blir veldig vanskelig om en ikke vet hvilke urenheter som er til stedet eller om de er veldig kjemisk lik analytten. Kompleksdannelse med EDTA er også pH sensitiv da det er en fireprotisk syre og kompleksdannelsen avhenger av protolysesteg. Dette krever egentlig høy pH, men siden kobberhydroksid er tungtløslig må en stabilisere pH innen et ganske lite vindu, så pH både for høy og for lav vil være en feilkilde.}


\subsubsection{Feilkilder for alternativ instrumentell metode}
\textit{ICP-MS vil være utsatt for matrikseffekter hvor andre elementer/isotoper med samme masse/ladningsforhold som analytten fører til positivt avvik fra faktisk konsentrasjon\cite{ICPMSGeneral2}. ICP-MS er også ikke fungerende på konsentrasjoner over rundt mikronivå\cite{ICPMSGeneral2}\cite{ICPMSGeneral}. Det vil si at løsningen i forsøket måtte fortynnes rundt 1000-10000 ganger som også introduserer flere mennesklige feil. Det kan også oppstå problemer med en del organisk matriale som ikke brytes fullstendig ned og kan feste seg i instumentet som vil redusere mengden stoff som kommer til detektoren og i verste fall ødelegge instrumentet.}

\subsubsection{Sammenligning opp mot alternative metoder}
\textit{Både jodometrisk titrering og komplekstitrering er begge veldig sensitive til urenheter i løsningen, og i stoffene som brukes under titeringen, samt mennesklige feil med over/undertitrering på grunn av fargeindikator. Avhengig av hvilke urenheter som er tilstede vil jodometri og komplekstitrering reagere ulikt på ulike forurensninger. Jodometri er mer påvirket av organiske stoffer, mens komplekstitrering er mer påvirket av metallioner. ICP-MS er mye mindre påvirket av mennesklige feil og mindre sensitiv til urenheter gitt da den i grunnen teller opp enkeltatomer. Den er også mer nøyaktig generelt så gitt at en har tilgang på instrumentet er det i de fleste tilfeller den bedre metoden.}
