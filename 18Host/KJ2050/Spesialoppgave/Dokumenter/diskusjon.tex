\section{Diskusjon}

\subsection{Standardisering av natriumtiosulfatløsning}
Standardiseringen av natriumtiosulfatløsningen viste at konsentrasjonen var lavere enn beregnet. Det betyr at det var mindre stoff i det utveide enn oppveid. Dette skyldes enten en feil i vekten, eller urent stoff, som vann i det tørkede stoffet.

De tre parallellene med kaliumjodatløsning ble tilsatt kaliumjodid på samme tid, men det tok lang tid før tilsetting av saltsyre. Etter tilsettingen av HCl ble løsningen umiddelbart titrert. I reaksjonen mellom KIO3 og KI (ligning \ref{eq:instill}) dannes det \chemfig{I_3^{-}}, altså jod i overskudd av jodid. Jod er flyktig (men ikke like mye i overskudd av jodid), og kan ha fordampet. Jod er det som reagerer som redoksreaksjon med natriumtiosulfat. Mindre jod vil føre til lavere titrervolum av natriumtiosulfat brukt som i beregninger vil gi en høyere konsentrasjon av natriumtiosulfat enn den faktiske. Reaksjon \ref{eq:instill} går mye raskere ved tilsetting av saltsyre, så det var antageligvis ikke mye som ble produsert før tilsetting av saltsyra (og etter ble det titrert straks), men dette kan ha påvirket resultatet av standardiseringen.

Byretta hadde en liten lekasje i hanen, noe som vil gi en viss usikkerhet i målingene. Dette vil gi et økt titrervolum, men det er uvisst i hvilken retning det vil slå ut for hver parallell.

Prøve 1 ble overtitrert, og derfor fjernet, så resultatene er kun basert på de to gjenstående parallellene. Dette øker usikkerheten og muligheten for utslag av tilfeldige feil.

\chemfig{CO_2}-fritt vann ble brukt i forsøket. Dette er fordi jodioner kan oksideres til \chemfig{I_2} i surt miljø (ligning \ref{eq:CO2_i_vann}), noe som er flyktig og vil skape feilkilder. 

Det var urenheter i kaliumjodatløsningen, mulig korkstøv fra \chemfig{CO_2}-fritt vannkolbene. Jod angriper kork, så dette vil muligens påvirke resultatene, men ikke på en skala som er i nærheten av utstyrets feilmargin. 

Hvis ventetiden fra innstilling til titrering er for lang vil tiosulfat kunne brytes ned og løsningen kan få en lavere konsentrasjon. Dette stemmer med standariseringen, da målt konsentrasjon er lavere enn den beregnet. Dette kan altså skyldes for lang tid mellom forberedelse av natriumtiosulfatløsning og titrering. Utslaget av dette kunne vært redusert ved å redusere tiden mellom instillingstitreringen av tiosulfatstandarden og tireringen av den ukjente prøven.

\subsection{Bestemmelse av kobberinnhold}
Hvis beregnet konsentrasjon for natriumtiosulfatløsninga fra standardiseringa er for høy gir det en høyere beregnet konsentrasjon for kobber også i resultatene fra titreringa. Det vil gi en høyere verdi for antall gram kobber. Dette stemmer ikke med vårt resultat, som er at verdien er lavere enn fasit. Lang ventetid før titrering kan også føre til tap av jod, som fører til en høyere verdi for kobber-innhold enn fasit.

Om beregnet konsentrasjon for natriumtiosulfat-løsning er for lav medfører det lavere mengde kobber beregnet i løsninga. Dette stemmer med fasit. Overtitrering ved innstilling av natriumtiosulfatløsninga, tidligere nevnt nedbryting av tiosulfat eller lekk byrette kan være årsak til en lavere verdi for kobber-innhold enn fasiten.

Prøven med en ukjent mengde kobbernitrat ($\chemfig{Cu} (\chemfig{NO_3})_2$) blandet med svovelsyre ble i forsøket oppvarmet på kokeplate for å drive ut svoveltrioksid (\chemfig{SO_3}) fra løsningen. Varmen på platen ble noe tilfeldig satt, og den anbefalte koketiden på ca ett minutt etter at hvit svoveltrioksidrøyk startet å framkomme ble kun omtrentlig overholdt. Kokingen fulgte dermed ikke nødvendigvis en ideell prosedyre, og kan, eksempelvis pga av manglende utdriving av \chemfig{SO_3}, ha bidratt til det noe lavere enn forventede sluttresultatet.


Surhetsgraden ble regulert uten tilgang til presist volumetrisk utstyr. Dette kan ha ført til at pH ikke var optimal før titreringen startet. Er løsningen for basisk kan hypojoditt dannes, og hypojoditt videre til jodid og jodat (se ligning \ref{eq:hypojodit}). Dette anses å potensielt kun være et problem om løsningen ikke surgjøres tilstrekkelig. pH må være nesten nøytral, men litt mer på sur side. Derfor nøytraliseres amoniakk med svovelsyre til fargeomslag og så ble omtrentlig en mL ekstra tilsatt.

\chemfig{I_2} og \chemfig{I_3^{-}} kan adsorbere på CuI fellingen. Det tilsettes derfor \chemfig{SCN^{-}} for å frigjøre jodet, men det vil likevel kunne skje at ikke all jod blir i løsningen. 

Da parallell nummer 1 ble titrert, var ikke byretten fylt helt opp, noe som medførte en etterfylling. Under etterfyllingen ble det en diskusjon rundt hva nøyaktig den avleste verdien var før ny titreringsløsning ble påfylt. Det vil derfor være en høyere usikkerhet tilknyttet det ene delvolumet som er benyttet i beregningen av det totalt brukte volumet.

I parallell 2 ble natriumtiosulfat tilsatt istedenfor stivelse. Dette ødela parallellen, og den ble forkastet. Dette øker den mulige spredningen av resultatene, og eventuelt forsterker bidraget fra den mistenkte økte usikkerheten i parallell nr. 1.

