\section{Eksperimentelt}

\subsection{Preparering og standardisering av natriumtiosulfatløsning}

Natriumtiosulfat (\chemfig{Na_2 S_2O_3}) (7,9939$\pm$0,0001\si{\gram}, vannfri) og kaliumjodat (\chemfig{KIO_3}) (1,3006$\pm$0,0001\si{\gram}) ble veid ut i hver sine beholdere og fortynnet til 500,0$\pm$0,5\si{\milli\liter} hver. 3 paralleller på 50,0$\pm$0,1\si{\milli\liter} kailumjodat ble pipettert ut i 250,0$\pm$0,3\si{\milli\liter}. Hver prøve ble tilsatt kaliumjodid (3$\pm$1\si{\gram}) og HCl (2\si{\milli\liter} 6\si{\molar}) og titrert umiddelbart mot tiosulfatløsningen. Stivelse (5$\pm$1\si{\milli\liter}) ble tilsatt som indikator da løsningen var svakt gul.


\subsection{Bestemmelse av kobber i ukjent prøve}

En ukjent prøve ble fortynnet til 250 mL. 3 paralleller av 50.00$\pm$0.035\si{\milli\liter} ukjent prøve ble overført til 250 mL erlenmeyerkolber og tilsatt 10$\pm$1\si{\milli\liter} konsentrert svovelsyre (\chemfig{H_2 SO_4}). Løsningen ble kokt til hvit \chemfig{SO_3} damp var godt synlig. Løsningen ble kjølt ned til romtemperatur og 20$\pm$3\si{\milli\liter} rent vann ble forsiktig tilsatt. Konsentrert ammoniakk ble tilsatt forsiktig til fargeendring fra grønn/turkis til dyp blå. 3\si{\molar} svovelsyre ble tilsatt til løsningen endret farge tilbake til grønnaktig og deretter ytterligere 1\si{\milli\liter}. Løsningen ble kjølt ned, tilsatt 4$\pm$0.5\si{\gram} kaliumjodid og titrert umiddelbart mot tiosulfatstandarden under omrøring til løsningen ble lys gulaktig. 5$\pm$1\si{\milli\liter} stivelse og 2$\pm$0.5\si{\gram} kaliumtiocyanat (\chemfig{KSCN}) ble tilsatt og det ble titrert forsiktig videre til fargeomslag.
