\documentclass{article}
\usepackage[margin=1.3in]{geometry}
\usepackage[norsk]{babel}
\usepackage[utf8]{inputenc}
\usepackage{graphicx}
\usepackage[section]{placeins}
\usepackage{amsmath}
\usepackage{amssymb}
\usepackage{mathtools}
\usepackage{esint}
\usepackage{harpoon}
\usepackage{mathtools}
\usepackage[labelfont=bf]{caption}
\usepackage{cancel}
\usepackage{pgfplots}
% \usepackage{pythontex}
% \usepackage[pyplot]{juliaplots}
\pgfplotsset{compat=1.7}
\usepackage{scalerel,stackengine}
\stackMath
\newcommand\reallywidehat[1]{%
\savestack{\tmpbox}{\stretchto{%
  \scaleto{%
    \scalerel*[\widthof{\ensuremath{#1}}]{\kern-.6pt\bigwedge\kern-.6pt}%
    {\rule[-\textheight/2]{1ex}{\textheight}}%WIDTH-LIMITED BIG WEDGE
  }{\textheight}% 
}{0.5ex}}%
\stackon[1pt]{#1}{\tmpbox}%
}

\newlength\figureheight
\newlength\figurewidth

\newcommand{\plotTikz}[3]{
	\setlength\figureheight{#3 cm}
	\setlength\figurewidth{#2 cm}
	\input{#1}
}

\setlength{\parindent}{0pt}

\newcommand{\partDiff}[2]{\frac{\partial #1}{\partial #2}}
\newcommand{\abs}[1]{\left| #1 \right|}
\newcommand{\paranth}[1]{\left( #1 \right)}
\newcommand{\bracket}[1]{\left[ #1 \right]}
\newcommand{\laplace}[1]{\mathcal{L}\paranth{#1}}
\newcommand{\invlaplace}[1]{\mathcal{L}^{-1}\paranth{#1}}
% \newcommand{\lim}[2]{\underset{\scalebox{0.8}{#1}}{\text{lim}} #2}

\begin{document}
\begin{center}
	\LARGE{\textbf{Øving 9}}
\end{center}


\section*{14.1}

\subsection*{6}


\plot{Figurer/f0.tikz}{6}{6}


\subsection*{11}


\begin{minipage}{0.4\textwidth}
	\plot{Figurer/f1.tikz}{6}{6}
\end{minipage}%
\hfill
\begin{minipage}{0.6\textwidth}
	\begin{gather*}
		z(t) = (1 + i) t + 3 i \qquad -1 \leq t \leq 1
	\end{gather*}
\end{minipage}%

\subsection*{22}


\begin{gather*}
	f(z) = \text{Re } z \quad \text{is not analytic, and integration has to be done by line integral:}
	\\
	y = 1 + \frac{1}{2} (x - 1)^2
	\qquad
	x = t, \quad y = 1 + \frac{1}{2} (t - 1)^2 \quad 1 \leq t \leq 3
	\\
	z = t + i \left(1 + \frac{1}{2} (t - 1)^2\right)
	\qquad
	\frac{d z}{d t} = 1 + i
	\\
	\int_{C}{f(z) dz} = \int_{1}^{3}{f(z(t)) \cdot \frac{dz}{dt} dt} = \int_{1}^{3}{t \cdot (1 + i) dt} = 4 + 4 i
\end{gather*}


\subsection*{25}


\begin{gather*}
	\int_1^i{z e^{z^2} dz} = \int_1^{-1}{\frac{1}{2} e^u du} = \left[\frac{1}{2} e^{u}\right]_1^{-1} = \frac{1}{2} e^{-1} - \frac{1}{2} e = -\sinh{1}
	\\
	u = z^2 \qquad 2 z dz
	\\
	\\
	z(t_1) = -t \quad -1 \leq t_1 \leq 0
	\qquad
	z(t_2) = i t_2 \quad 0 \leq t_2 \leq 1
	\\
	\\
	\int_C{f(z) dz} = \int_{-1}^0{f(z(t_1)) \frac{dz}{dt_1} dt_1} + \int_{0}^1{f(z(t_2)) \frac{dz}{dt_2} dt_2}
	=
	\\
	\int_{-1}^0{-t_1 e^{t_1^2} (-1) dt_1} + \int_{0}^1{i t_2 e^{-t_2^2} i dt_2}
	=
	\frac{1}{2} - \frac{1}{2} e - \frac{1}{2} + \frac{1}{2} e^{-1}
	=
	\\
	\frac{1}{2} e^{-1} - \frac{1}{2} e = -\sinh{1}
\end{gather*}


\subsection*{29}


\begin{gather*}
	f(z) = \text{Im } z^2 = 2 i x y \qquad \text{not analytic}
	\\
	z(t_1) = t_1 \quad 0 \leq t_1 \leq 1
	\quad
	z(z_2) 1 - t_2 + i t_2 \quad 0 \leq t_2 \leq 1
	\quad
	z(t_3) = -i t_3 \quad -1 \leq t_3 \leq 0
	\\
	f(z(t_1)) = 0
	\qquad
	f(z(t_2)) = 2 i (1 - t_2) \cdot t_2
	\qquad
	f(z(t_3)) = 0
	\qquad
	\frac{d z}{d t_2} = -1 + i
	\\
	\\
	\int_C{f(z) dz} = \int_0^1{2 i (t_2 - t_2^2) (-1 + i)}
	=
	-(2 + 2 i) \int_0^1{(t_1 - t_2^2) dt_2}
	=
	\\
	-(2 + 2 i) \left[\frac{1}{2} t_2^2 - \frac{1}{3} t_2^3\right]_0^1
	=
	-(2 + 2 i) \left(\frac{1}{2} - \frac{1}{3}\right) = -\frac{1}{3} - \frac{1}{3} i
\end{gather*}


\section*{14.2}

\subsection*{4}


No, the function cannot be analytic in the annulus since the integral over the border would be 6 - 3 = 3 which is not zero.


\subsection*{13}


\begin{gather*}
	f(z) = \frac{1}{z^4 - 1.2} \qquad \text{ Analytic for all } z \in \mathbb{C} \text{\textbackslash} \left\{\pm 1.2^{1/4}, \pm i 1.2^{1/4}\right\}
	\\
	z(t) = e^{i t} \qquad 0 \leq t \leq 2 \pi
	\qquad
	\frac{dz}{dt} = i e^{i t}
	\\
	\int_0^{2 \pi}{\frac{i e^{i t}}{e^{4 i t}} - 1.2} dt
	=
	\int_{-0.2}^{-0.2}{\frac{1}{u} \frac{1}{4 e^{3 i t}} du}
	=
	\frac{1}{4} \int_{-0.2}^{-0.2}{\frac{1}{u^{\frac{7}{4}}} du} = 0
	\\
	u = e^{4 i t} - 1.2 \quad du = 4 i e^{4 i t} dt \quad e^{3 i t} = \left(e^{4 i t}\right)^{\frac{3}{4}} = u^{\frac{3}{4}}
	\\
	\text{Cauchy's integral theorem applies} \Rightarrow f(x) \text{ is analytic in the unit circle}
\end{gather*}


\subsection*{22}


\begin{gather*}
	f(z) = \text{Re } z
	\\
	z(t_1) = t_1 \quad -1 \leq t_1 \leq 1 \qquad z(t_2) = e^{i t_2} \quad 0 \leq t_2 \leq \pi
	\\
	f(z(t_1)) = t_1 \quad \frac{dz}{dt_1} = 1 \qquad f(z(t_2)) = \cos{t_2} \quad \frac{dz}{dt_2} = i e^{i t_2}
	\\
	\\
	\int_C{f(z) dz} = \int_{-1}^{1}{t_1 dt_1} + \int_0^\pi{\cos{t_2} i e^{i t_2} dt_2} =
	\\
	0 + \int_0^\pi{i \cos{t_2} (\cos{t_2} + i \sin{t_2}) dt_2}
	=
	\int_0^\pi{(i \cos^2{t_2} - \cos{t_2} \sin{t_2}) dt_2}
	=
	\\
	\int_0^\pi{\left(\frac{1}{2} (1 + \cos{2 t_2}) - \frac{1}{2} \sin{2 t_2}\right) dt_2}
	=
	\int_0^\pi{\frac{1}{2}dt_2} = \frac{\pi}{2} \neq 0
\end{gather*}


\subsection*{24}


\begin{gather*}
	\oint_C{\frac{dz}{z^2 - 1}} = \oint_C{\left(\frac{\frac{1}{2}}{z - 1} - \frac{\frac{1}{2}}{z + 1}\right) dz}
	=
	\oint_{C_1}{\frac{\frac{1}{2}}{z - 1} dz} - \oint_{-C_2}{\frac{\frac{1}{2}}{z + 1} dz}
	=
	\\
	2 i \pi g(-1) - 2 i \pi g(1) = 0, \qquad g(z) = \frac{1}{2}
\end{gather*}


\section*{17.1}


\subsection*{10}


\plot{Figurer/f2.tikz}{7}{7}
\plot{Figurer/f3.tikz}{7}{7}



\end{document}
