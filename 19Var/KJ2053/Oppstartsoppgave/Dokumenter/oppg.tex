\title{\textbf{Oppstartsoppgave}}
\maketitle

\section*{Del A:}

\subsection*{1.}

Ordet kromatografi kommer av de greske ordene kromos og grafos, som betyr henholdsvis farge og tegning, altså betyr kromatografi direkte "fargetegning". Dette kommer av teknikkens opprinnelse til å seperere plantestoffer som ble seperert i forskjellige farger.
\smallskip
\\
Generelt er kromatografi en seperasjonsmetode basert på hvordan stoffer fordeler seg mellom to faser. Dette kan for eksemplel være løselighetsforskjeller mellom to væsker. I praksis gjøres dette alltid ved å ha en fase som står i ro, kjent som den stasjonære fasen, og en fase som beveger seg over/gjennom den stasjonære fasen som da er den mobile fasen. Den stasjonære fasen er generelt et fast eller flytende stoff, mens den mobile fasen er væske eller gass. Stoffer vil da bli flyttet gjennom den stasjonære fasen av den mobile fasen med en fart som avhenger av hvor stor andel av tiden stoffet oppholder seg i hver av fasene. Et stoff som har stor affinitet til den mobile fasen og kun oppholder seg i den vil forflyttes (elueres) i samme hastighet som mobilfasen, mens et stoff med stor affinitet til stasjonærfasen vil bli eluert mye tregere da den tilbringer mye mer tid i stasjonærfasen. Andelen av tiden et stoff oppholdes i stasjonærfasen kalles retensjonsfaktoren. Ulike kromatografiske metoder bruker gjerne ulike måter å måle resultatet på. For eksempel TLC er en konstant tid metode, hvor prosessen avsluttes etter et gitt tidspunkt og elueringslengdene måles. Stoffer med høyere retensjonsfaktor vil da ha en lavere elueringslengde. I f.eks HPLC sendes mobilfasen gjennom en tettpakket kolonne og elueringstiden måles, dvs hvor lang tid stoffene bruker gjennom kolonnen. Høyere retensjonsfaktor her vil da gi høyere tid. Dimensjonene på ulike stasjonærfasekollonner avhenger av hvilke stasjonærfase og mobilfase en bruker. For eksempel en kapillærkolonne til gass-væskekromatografi er typisk mindre enn en millimeter tykk og mellom 15 og 30 meter lang (kveilet opp). Kromatografi kan brukes til preperativ seperasjon av stoffer, men dette er ofte tungvindt og kromatografi brukes stort sett som en analytisk seperasjonsmetode.


\subsection*{2.}

Har en del egen erfaring med TLC fra lab i organisk kjemi (KJ1020) hvor vi sjekker for tilstedeværelse av urenheter i produktet. Dette gjorde vi med å applisere små mengder av analytten, samt referanser av rent stoff og mistenkte urenheter på en tynn silikaplate som er stasjonærfasen. Silikaplaten ble satt ned i en mobilfase av petroleumseter og aceton som trekker opp i platen av kapillærkrefter. Platen tas ut av mobilfasen når væskefronten nærmer seg toppen og lufttørkes og retensjonslengdene måles.
\smallskip
\\
Andre kjente metoder er GC, LC, HPLC og superkritisk fluid kromatografi.
\\
I GC (Gas Chromatography) er mobilfasen en gass og stasjonærfasen flytende, ofte derivater av polysiloksaner som danner et tynt lag på innsiden av en lang tynn kapillærkolonne. I LC (Liquid Chromatography) er mobilfasen en væske som trykkes gjennom en kolonne med pakket fast stasjonærfase. HPLC (High Performance Liquid Chromatography) ligner mye på vanlig LC, men kjøres på mye høyere trykk i mye tettere kolonner ove mye lengre tid og gir mye bedre seperasjon. Superkritisk fluid kromatografi er som LC, men mobilfasen er en superkritisk fluid som for eksempel superkritisk \chemfig{CO_2}.


\subsection*{3.}

Jeg går anvendt teoretisk kjemi og er nok ikke en av de som kommer til å bruke kromatografiske metoder på en veldig regulær basis, men jeg kan få en del bruk for å verifisere teoretisk beregnede affiniteter og løseligheter. Det kan også brukes for å sjekke for urenheter i stoffer som gjøres andre eksperimenter og analyser på.


\section*{Del B:}

\subsection*{4.}

GC: Gas Chromatography. Som beskrevet i punkt 2, gass som mobilfase og væske som stasjonærfase.
\\
HPLC: High Performance Liquid Chromatography. Også ofte kalt High Pressure på grunn av det høye trykket det kjøres på. Det er som beskrevet i punkt 2, væske som mobilfase, solid stasjonærfase, høyt trykk, tett pakket stasjonærfase, lang tid og god seperasjon.
\\
TLC: Thin Layer Chromatography. Også forklart i punkt 2, væske som mobilfase og en tynn plate av silika som mobolfasen trekkes gjennom av kapillærkrefter. Prosessen stoppes ved et gitt tidspunkt og retensjonslengdene måles.
\\
SEC: Size Exclusion Chromatography. Porøs solid stasjonærfase hvor retensjon bestemmes av størrelsen til partikklene hvor større partikkler sitter seg fast i flere porer og har høyere retensjonsfaktor.
\\
SFC: Super-critical Fluid Chromatography. Også nevnt litt i punkt 2. Tilsvarende LC, men med superkritisk fluid som mobilfase.


\subsection*{5.}

Kvalitativ analyse ser på kvalitenene av noe, ofte for å finne ut hvilke stoffer en har, men ikke hvor mye.
Kvantitativ analyse ser på kvantiteten av noe, dvs. hvor mye en har av noe.


\subsection*{6.}

En injektor injekterer stoff inn i mobilfasen, dvs putter analytten inn i mobilfasen. En detektor detekterer stoffene som kommer ut av kolonnen og sender tisdpunkt og stoffmengde til en datamaskin som logger resultatene.


\subsection*{7.}

Varians er gjennomsnittet av kvadratavvikene til datapunktene fra middelverdien som forteller noe om spredningen i datasettet. Varansen har enhet som er kvadrert av enhetene til datapunktene så for å få spredningen i en enhet som reflekterer spredningen i rett enhet tar en kvadratrot av varansen som er standardavviket.
\begin{gather*}
	Var(X) = \frac{1}{n - 1} \sum_{i = 1}^n{\left(x_i - \bar{X}\right)}
	\\
	SD(X) = \sqrt{Var(X)}
\end{gather*}


\subsection*{8.}

\textit{n}-heksan:
\hfill
\smallskip
\\
\schemestart
	\chemfig{-[-30]-[30]-[-30]-[30]-[-30]}
\schemestop
\hfill
\bigskip
\\
Aceton:
\hfill
\smallskip
\\
\schemestart
	\chemfig{-[30](=[90]O)-[-30]}
\schemestop
\hfill
\bigskip
\\
Dietyleter:
\hfill
\smallskip
\\
\schemestart
	\chemfig{-[-30]-[30]O-[-30]-[30]}
\schemestop
\hfill
\bigskip
\\
Diklormetan (DCM):
\hfill
\smallskip
\\
\schemestart
	\chemfig{C(-[150]Cl)(-[30]Cl)(-[-150]H)(-[-30]H)}
\schemestop
\hfill
\bigskip
\\
Kloroform (Triklormetan):
\hfill
\smallskip
\\
\schemestart
	\chemfig{C(-[150]Cl)(-[30]Cl)(-[-150]Cl)(-[-30]H)}
\schemestop
\hfill
\bigskip
\\
Toluene:
\hfill
\smallskip
\\
\schemestart
	\chemfig{-[-90]!{Ph}}
\schemestop
\hfill
\bigskip
\\
Etylacetat:
\hfill
\smallskip
\\
\schemestart
	\chemfig{-[30](=[90]O)-[-30]O-[30]-[-30]}
\schemestop
\hfill
\bigskip
\\
Metanol:
\hfill
\smallskip
\\
\schemestart
	\chemfig{C(-[150]H)(-[30]OH)(-[-150]H)(-[-30]H)}
\schemestop
\hfill
\bigskip


\subsection*{9.}

"Likt løser likt" refererer til at stoffer med lik polaritet løser/løses godt i hverandre. Dette ser vi for eksempel i at olje og vann ikke er løselig i hverandre. Dette konsepetet er meget viktig for kromatografi da kromatografi i stor grad handler om løselighet i forskjellige stoffer.
