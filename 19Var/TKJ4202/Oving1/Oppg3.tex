## b)

$$ P V = n R T $$

$$ p_1 = \frac{n R T_1}{V_1} = \frac{0.0821 \cdot 298}{5} = 4.89 atm $$
$$ p_2 = \frac{n R T_1}{V_2} = \frac{0.0821 \cdot 298}{2.5} = 9.79 atm $$
$$ T_2 = \frac{p_1 V_2}{n R} = \frac{4.81 \cdot 2.5}{0.0821} = 146.5 K $$

## c)

Trinn 1:
\
$$
q = R T \ln{\frac{P_1}{P_2}} = 8.3145 \cdot 298 \cdot \ln{\frac{4.89}{9.79}}
= -1.72 kJ
$$
$$ w = -q = 1.72 kJ $$
$$ \Delta U = \Delta H = 0 $$

Trinn 2:
\
$$ q = C_V\ (T_2 - T_1) = \frac{3}{2} \cdot 8.3145 (146.5 - 298) = -1.89 kJ $$
$$ w = 0 kJ $$
$$
\Delta U = q = -1.89 kJ
$$
$$
\Delta H = C_p\ (T_2 - T_1) = \frac{5}{2} \cdot 8.3145 (146.5 - 298) = -3.15 kJ
$$

Totalt:
\
$$ q = -1.72 kJ - 1.89 kJ = - 3.61 kJ $$
$$ w = 1.72 kJ + 0 kJ = 1.72 kJ $$
$$ \Delta U = 0 kJ - 1.89 kJ = -1.89 kJ $$
$$ \Delta H = 0 kJ - 3.15 kJ = -3.15 kJ $$

## d)

$$ q = C_p\ (T_2 - T_1) = \frac{5}{2} \cdot 8.3145 (146.5 - 298) = -3.15 kJ $$
$$ w = R (T_1 - T_2) = 8.3145 (298 - 146.5) = 1.26 kJ $$
$$ \Delta U = C_V (T_2 - T_1) = -1.89 kJ $$
$$ \Delta H = C_P (T_2 - T_1) = -3.15 kJ $$

Her ser vi at $\Delta U$ og $\Delta H$ er lik for de to prosessene som gir mening da de er tilstandsfunksjoner og de ender opp på samme sluttilstand, men da de har forskjellige veier til denne tilstanden vil ikke utført arbeid og varme nødvendigvis være lik, som vi ser at de ikke er.
