## b)

$$ P V = n R T $$
$$ T = \frac{PV}{nR} $$
Trinn 1:
$$ T = \frac{2\ atm ⋅ 10\ L}{1\ mol ⋅ 0.0821\ L\ atm\ K^{-1}\ mol^{-1}}
= 243.6\ K
$$

Etter trinn 2:
$$
T = \frac{20\ atm ⋅ 10\ L}{1\ mol ⋅ 0.0821\ L\ atm\ K^{-1}\ mol^{-1}} = 2436\ K
$$

## c)

$$ \Delta U = c_V ⋅ \Delta T $$
$$ \Delta H = c_P ⋅ \Delta T $$


Trinn 1:
\
Isoterm $⇒\ ΔT = 0 ⇒ ΔU = ΔH = 0$
$$
q = R T \ln{\frac{P_1}{P_2}}
= 8.3145 ⋅ 243.6 ⋅ \ln{\frac{2}{20}} = -4.67 kJ
$$
$$ΔU = q + w ⇒ w = -q = - (- 4.67 kJ) = 4.67 kJ$$


Trinn 2:
\
$$ \Delta U = \frac{3}{2} R ⋅ (2436 - 243.6) = 27.3 kJ $$
$$ \Delta H = \frac{5}{2} R ⋅ (2436 - 243.6) = 45.6 kJ $$
isobar:
$$ q = \Delta H ⇒ q = 45.6 kJ $$
$$ w = R\ (243.6 - 2436) = -18.2 kJ $$


Trinn 3:
\
$$ \Delta U = \frac{3}{2} R ⋅ (243.6 - 2436) = - 27.3 kJ $$
$$ \Delta H = \frac{5}{2} R ⋅ (243.6 - 2436) = - 45.6 kJ $$
$$ q = \Delta U = - 27.3kJ $$
$$ w = \int_{V_1}^{V_2}{P(V) dV} = 0 $$


Totalt:
\
$$ \Delta U = 0 + 27.3 kJ - 27.3 kJ = 0 kJ $$
$$ \Delta H = 0 + 45.6 kJ - 45.6 kJ = 0 kJ $$
$$ q = -4.67 kJ + 45.6 kJ - 27.3 kJ = 13.6 kJ $$
$$ w = 4.67 kJ - 18.2 kJ + 0 kJ = -13.6 kJ $$