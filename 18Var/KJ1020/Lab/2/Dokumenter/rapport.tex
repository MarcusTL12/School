\begin{center} % Tittelen på dokumentet, sentrert, stor og i bold skrift
\LARGE{\textbf{Eksperiment 2 - Omkrystallisering av ekstrahert acetylsalisylsyre}}
\end{center}
[Alt som står i firkantklammer er instruksjoner/kommentarer som skal fjernes, eller erstattes, unntatt enhetene i datatabeller. Se ellers vedlegg i oppgaveheftet for organisk-lab om oppsett av rapport.]
\section*{Sammendrag}
Acetylsalisylsyre ble ekstrahert fra aspirin-tabletter og isolert ved omkrystallisering i etylacetat. Renheten til det endelige produktet ble evaluert ved bestemmelse av smeltepunktsintervall. Råproduktet var ekstrahert i [FYLL INN HER]\% utbytte, og acetylsalisylsyre ble isolert i [FYLL INN HER]\% utbytte. Smeltepunktsanalysen viste at det endelige produktet er [FYLL INN HER: konklusjon om renhet].


\section{Teori}
En løsning der konsentrasjonen av en ellers fast forbindelse er høyere enn løseligheten til forbindelsen, kalles for en overmettet løsning\cite{raymond2013general}. Dette kan oppnås enten ved å øke konsentrasjonen (dampe av løsningsmiddelet), eller redusere løseligheten i løsningen. Ved veldig høy overmetting, eller i nærvær av en skarp overflate, enten fra en tilsatt krystall eller en mikroskopisk ripe i glasset, vil den løste forbindelsen felle ut som en krystall.

\subsection{Omkrystallisering}
Ved å løse en uren krystallinsk forbindelse i en minimal mengde varmt løsningsmiddel, og deretter la temperaturen sakte synke, vil løsningen gradvis bli overmettet; Den ønskede forbindelsen vil felle sakte ut som krystaller, mens urenhetene forblir i løsningen\cite{harwood1999experimental}. Denne metoden er kjent som omkrystallisering, og er en av de vanligste rensemetodene i et organisk laboratorium. Metoden er også mye brukt i kjemisk industri, da den fungerer like godt i stor skala. 
Omkrystallisering består av 5 steg: Oppløsning, filtrering, krystallisering, isolering og tørking.

\subsubsection{Oppløsning og filtrering}
Den viktigeste egenskapen til et løsningsmiddel i en omkrystallisering er at løseligheten til forbindelsen må ha størst mulig differanse i varmt og kaldt løsningsmiddel\cite{harwood1999experimental}. Det må være flyktig nok til at krystallene kan tørkes etter at de er filtrert, samtidig som det bør ha et høyt kokepunkt slik at temperaturforskjellen blir tilstrekkelig stor; Det bør helst ikke kjøles ned under 0 \si{\celsius}, fordi vann fra atmosfæren da vil kondensere på krystallene. 

I utvelgingen av løsningsmiddel er det vanlig å følge regelen likt løser likt. For krystallinske hydrokarboner kan f.eks. pentan, sykloheksan eller toluen brukes, mens forbindelser med karbonylgrupper kan omkrystalliseres i etylacetat eller aceton\cite{harwood1999experimental}. Salter og veldig polare hydrokarboner omkrystalliseres i vann; Vanlig sukker renses i industriell skala med vann\cite{tro2011chemistry}.

Oppløsningen utføres ved å blande krystallene med en for liten mengde løsningsmiddel, og varme dette til koking med tilbakeløp. Deretter tilsettes litt og litt løsningsmiddel til alt er løst ved denne temperaturen. Hvis ikke et enkelt løsningsmiddel passer til kriteriene, kan blandinger av løsningsmiddel brukes\cite{harwood1999experimental}.  Da løses forbindelsen i en minimal mengde av et løsningsmiddel med for høy løselighet, deretter tilsettes et løsningsmiddel med for lav løselighet dråpevis til løsningen blir tåkete og ugjennomsiktig. 

Filtrering av den varme løsningen kan utføres dersom det er urenheter tilstede, som er mindre løselige enn den ønskede forbindelsen.


\subsubsection{Krystallisering}
Den vanligste prosedyren for krystallisering er å la løsningen stå i romtemperatur og kjøle seg ned\cite{harwood1999experimental}. Dersom dette ikke er tilstrekkelig, kan det kjøles videre i isbad, tilsettes en podekrystall og/eller skrapes med en glasstav i kolben. Podekrystallen og skrapingen har samme hensikt; De skaper en overflate der de første molekylene kan krystallisere. Siden de intermolekylære kreftene er sterkere mellom like molekyler vil hele krystallen bygges opp hovedsakelig av like molekyler, men hvis krystalliseringen skjer for fort kan urenheter bli ''fanget`` i krystallen. 

\subsubsection{Isolering og tørking}
Når krystallene er dannet må de isoleres fra den gjenværende løsningen (moderluten). Dette kan gjøres på ulike måter, avhengig av skalaen, men den vanligste metoden i et organisk laboratorium er filtrering over vakuum. Når krystallene er filtrert kan de skylles med litt kaldt løsningsmiddel, for å bli kvitt urenheter som har blitt igjen utenpå krystallene. Dersom en stor andel av forbindelsen fortsatt er løst i moderluten (filtratet) kan denne oppkonsentreres ved inndamping av løsningsmiddelet og prosessen gjentas.

De filtrerte krystallene tørkes rimelig effektivt ved å la de ligge på filteret med vakuum under i 2–3 minutter. Dersom de fortsatt ikke er helt tørre, kan de tørkes ytterligere under vakuum i en lukket beholder.

\subsection{Smeltepunktsbestemmelse}
Smeltepunktet for en forbindelse er definert som den temperaturen der væskefasen og den faste fasen til forbindelsen begge eksisterer i likevekt\cite{raymond2013general2nd}. I praksis blir smeltepunkt målt som et intervall fra den første dråpen dannes til de siste krystallene er smeltet, og i de fleste tilfeller er et kort intervall en god indikasjon på at krystallene er rene\cite{harwood1999experimental2nd}. Dette kommer av at en mindre urenhet vil bryte opp den ellers systematiske krystallstrukturen, og redusere de intermolekylære kreftene som holder krystallen sammen. Unntaket fra indikasjonen er når sammensetningen danner en eutektisk blanding\cite{harwood1999experimental2nd,snl}. Ved denne sammensetningen vil smeltepunktet være lavere enn for begge de rene forbindelsene, men intervallet vil fortsatt være kort

\section{Fysikalske data}
Tabell \ref{tab:fysdata} viser relevante fysikalske data til forbindelser brukt i dette forsøket. 
\begin{table}[ht!]
	\begin{center}
		\caption{Fysikalske data for aktuelle forbindelser \cite{CRC}}
		\label{tab:fysdata}
		\begin{tabular}{l c c}
		\toprule
		Forbindelse & Kokepunkt[\si{\celsius}] & Smeltepunkt[\si{\celsius}]\\
		\midrule
		Acetylsalisylsyre & - & 136\\
		Etylacetat & 77 & -\\
		\bottomrule
		\end{tabular}
	\end{center}
\end{table}



\section{Eksperimentelt}
\subsection{Ekstraksjon}
Aspirin (4 tabletter) ble knust til pulver i en morter, og overført til en erlenmeyerkolbe.  Etylacetat (\SI{10}{\milli\liter}) ble brukt til å skylle morteren for å få med alt. En røremagnet ble tilsatt og kolbeåpningen ble dekket til med aluminiumsfolie. Det ble varmet opp under omrøring til kondens ble observert i kolbehalsen. Blandingen ble dekantert gjennom et foldefilter over i en rundkolbe. Oppvarmingen ble gjentatt 2 ganger med etylacetat (2 x \SI{10}{\milli\liter}), men den siste gangen ble alt innholdet i kolben overført til foldefilteret. Filtratet ble dampet inn under redusert trykk til råproduktet.
\subsection{Omkrystallisering}
Til rundkolben med råproduktet ble det tilsatt etylacetat ([FYLL INN HER] \si{\milli\liter}) og et par kokesteiner. En kjøler ble montert loddrett opp av rundkolben som ble varmet til koking i en varmekappe. [Det ble tilsatt ytterligere etylacetat ([FYLL INN HER: eller fjern hele setningen hvis dette ikke ble gjort.] \si{\milli\liter}) gjennom kjøleren for å løse alt råproduktet.] Da alt var løst ble varmekilden fjernet og kolben fikk komme sakte til romtemperatur, før den ble kjølt ytterligere i et isbad. Det ble brukt en glasstav til å skrape i bunnen av rundkolben for å starte utkrystalliseringen. Krystallene ble filtrert og tørket over en Büchnertrakt, før de ble overført til et urglass. Smeltepunktsintervallet til krystallene ble bestemt ved hjelp av et smeltepunktsapparat tre ganger. 



\section{Resultater}

\section{Diskusjon}
