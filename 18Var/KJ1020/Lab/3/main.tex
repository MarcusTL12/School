%%%%%%%%%%%%%%%%%%%%%%%%%%%%%%%%%%%%%%%%%%%%%%%%%%%%%%%%%%%%%%%%%%%%%%
%%%%%%%%%%%%%%%%%%%%%%%%%%%%%%%%%%%%%%%%%%%%%%%%%%%%%%%%%%%%%%%%%%%%%%
%%%%%%%%%%%%%%%%Pakker og instillinger for dokumentet%%%%%%%%%%%%%%%%%

\documentclass[pdftex, 10pt, norsk, a4paper, twoside]{article} 
\usepackage[margin=1.3in]{geometry}							
\usepackage[norsk]{babel}%Legger til norske bokstaver		
\selectlanguage{norsk}%Får latex til å bruke norsk på steder der latex selv skriver ting. F.eks. Figur istedet for Figure	
\usepackage[T1]{fontenc} %Definerer hvilken font som skal brukes. For spesielt interreserte er dette en 8-bit encoding med fonter som bruker 256 glyfer							
\usepackage[utf8]{inputenc}%Tillater deg å skrive norske bokstaver direkte fra tastaturet, i steden for å bruke kommandoer.		
\usepackage{graphicx}% Nødvendig for å legge til bilder osv. 
\usepackage[section]{placeins}
\usepackage{cancel}% for å kunne "stryke over" ting i ligninger feks \frac{\bcancel{b}}{\bcancel{b}} = 1
\usepackage{fmtcount}
\usepackage{siunitx} % Lar deg bruke kommandoene \SI{tall}{\enhet} og \si{\enhet}. feks \si{\milli\gram\per\liter}. Dette er en veldig fin måte for å få konsistene enheter på. 
\usepackage{booktabs}% lar deg bruke \toprule, \midrule og \bottomrule i tabeller for å lage pene tabeller.								
\usepackage{multirow}% Lar deg lage rader inni radene i en tabell.
\usepackage[pdftex,bookmarks,breaklinks]{hyperref}
\usepackage{parskip}% Gjør at latex tolker paragrafer og ikke bare ignorerer "white space"
\usepackage{rotating} %Gir deg en rekke muligheter for å rotere ting
\usepackage	{amsmath , amsfonts , amssymb}							
\usepackage[hang,footnotesize,bf]{caption}% Gjør at captions under figurer og over tabeller blir pene.						
\usepackage[version=3]{mhchem} % Tillater deg å skrive kjemiske formler
\usepackage{epstopdf}% Omgjør .eps-filer til .pdf filer. Denne pakken er genial når man bruker chemdraw der man kan eksportere som eps. 
\usepackage{textcomp,gensymb}
\usepackage{float}% Tillater å bruke stor H i floats, dette kalles ofte for float of doom og er skjeldent anbefalt.
\usepackage[super,square]{natbib}% Denne pakken er for bibliografien, super og square gjør at sitatet opphøyen og inlemmes av firkantparanteser. 
\usepackage{fancyhdr}% Lar deg lage den fine headeren på toppen av siden.
\pagestyle{fancy}
\newcommand\EatDot[1]{} % Fjerner punktum på slutten av referansen.
%%%%%%%%%%%%%%%%%%%%%%%%%%%%%%%%%%%%%%%%%%%%%%%%%%%%%%%%%%%%%%%%%%%%%%


%%%%%%%%%%%%%%%%%%%%%%%%%%%%%%%%%%%%%%%%%%%%%%%%%%%%%%%%%%%%%%%%%%%%%%
%Setter noen instillinger for å gjøre ting pent :) 
\setlength\headheight{22.5pt}	
\floatstyle{plaintop}													
\restylefloat{table}													
\floatstyle{plain}														
\restylefloat{figure}													
\numberwithin{equation}{section}
\numberwithin{figure}{section}
\numberwithin{table}{section}
%%%%%%%%%%%%%%%%%%%%%%%%%%%%%%%%%%%%%%%%%%%%%%%%%%%%%%%%%%%%%%%%%%%%%%


%%%%%%%%%%%%%%%%%%%%%%%%%%%%%%%%%%%%%%%%%%%%%%%%%%%%%%%%%%%%%%%%%%%%%%
%Lager headeren
\lhead{Kari Nordmann \\ Gruppe 1, plass 17A}			
\chead{}																
\rhead{TMT4122 -- Organisk kjemi laboratoriekurs  \\ \today}	
\lfoot{}
\cfoot{\thepage}												
\rfoot{}
%%%%%%%%%%%%%%%%%%%%%%%%%%%%%%%%%%%%%%%%%%%%%%%%%%%%%%%%%%%%%%%%%%%%%%
%Denne kodebiten lager en egen funksjon som heter \signature{}{} med input sted og navn. Den brukes litt lengre ned i koden
\newcommand{\signature}[2]{
\begin{minipage}[t]{0.9\textwidth}
\vspace{1cm}

    #1, \today
\vspace{1.5cm}

\noindent
\begin{tabular}{cc}
    \rule{6cm}{1pt} & \hspace{2cm} \\
    #2 & 
\end{tabular}
\vspace{1cm}
\end{minipage}
}
%%%%%%%%%%%%%%%%%%%%%%%%%%%%%%%%%%%%%%%%%%%%%%%%%%%%%%%%%%%%%%%%%%%%%%




\begin{document}

% \begin{center} % Tittelen på dokumentet, sentrert, stor og i bold skrift
\LARGE{\textbf{Eksperiment 3 - Ekstraksjon}}
\end{center}
[Alt som står i firkantklammer er instruksjoner/kommentarer som skal fjernes, eller erstattes, unntatt enhetene i datatabeller. Se ellers vedlegg i oppgaveheftet for organisk-lab om oppsett av rapport.]
\section*{Sammendrag}
I dette eksperimentet ble en blanding av benzosyre og bifenyl separert ved væske-væske ekstraksjon. Hensikten var å isolere forbindelsene og bestemme det opprinnelige blandingsforholdet. Benzosyre ble isolert i [FYLL INN HER]\% utbytte, og bifenyl ble isolert i [FYLL INN HER]\% utbytte. Dette ga et blandingsforhold på [FYLL INN HER]. Smeltepunktsanalyser viste at begge produktene er [FYLL INN HER: konklusjon om renhet].




\section{Teori}


\section{Fysikalske data}
Tabell \ref{tab:fysdata} viser relevante fysikalske data til forbindelser brukt i dette forsøket. 
\begin{table}[ht!]
	\begin{center}
		\caption{Fysikalske data for aktuelle forbindelser \cite{CRC}}
		\label{tab:fysdata}
		\begin{tabular}{l c c}
		\toprule
		Forbindelse & Smeltepunkt[\si{\celsius}]\\
		\midrule
		Bifenyl & 69\\
		Benzosyre & 122\\
		\bottomrule
		\end{tabular}
	\end{center}
\end{table}



\section{Eksperimentelt}

\section{Resultater}

\section{Diskusjon}
 importerer fila som ligger i mappen Dokumenter og henter latexkoden derfra. Dette er for å ikke blande tekst og masse kode som er i dette dokumentet
\begin{center} % Tittelen på dokumentet, sentrert, stor og i bold skrift
\LARGE{\textbf{Eksperiment 3 - Ekstraksjon}}
\end{center}
[Alt som står i firkantklammer er instruksjoner/kommentarer som skal fjernes, eller erstattes, unntatt enhetene i datatabeller. Se ellers vedlegg i oppgaveheftet for organisk-lab om oppsett av rapport.]
\section*{Sammendrag}
I dette eksperimentet ble en blanding av benzosyre og bifenyl separert ved væske-væske ekstraksjon. Hensikten var å isolere forbindelsene og bestemme det opprinnelige blandingsforholdet. Benzosyre ble isolert i [FYLL INN HER]\% utbytte, og bifenyl ble isolert i [FYLL INN HER]\% utbytte. Dette ga et blandingsforhold på [FYLL INN HER]. Smeltepunktsanalyser viste at begge produktene er [FYLL INN HER: konklusjon om renhet].




\section{Teori}


\section{Fysikalske data}
Tabell \ref{tab:fysdata} viser relevante fysikalske data til forbindelser brukt i dette forsøket. 
\begin{table}[ht!]
	\begin{center}
		\caption{Fysikalske data for aktuelle forbindelser \cite{CRC}}
		\label{tab:fysdata}
		\begin{tabular}{l c c}
		\toprule
		Forbindelse & Smeltepunkt[\si{\celsius}]\\
		\midrule
		Bifenyl & 69\\
		Benzosyre & 122\\
		\bottomrule
		\end{tabular}
	\end{center}
\end{table}



\section{Eksperimentelt}

\section{Resultater}

\section{Diskusjon}

\vfill % Gjør at signaturen havner nederst på siden, uavhengig av hvor mye tekst det er på den. Dette er en god praksis
\signature{Trondheim}{Kari Nordmann} % lager en signaturlinje.

\bibliographystyle{unsrtnat}%Definerer bibliografistil
\bibliography{bibliografi}%navnet i krøllparantesene er navnet på .bib filen til bibliografien
%Vedleggene skal alltid komme etter referanselisten og skal plaseres i en egen fil, på en ny side. Derfor brukes \newpage
\newpage 
\section*{Vedlegg A}

[Eventuelle vedlegg skal inn her.]


\end{document} 